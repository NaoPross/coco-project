% vim: ts=2 sw=2 et spell:
\documentclass[xetex, smaller, aspectratio=43]{beamer}
\usepackage[english]{babel}
\usetheme{hsr}

% for draft, add [label=current] to frame
% \includeonlyframes{current}

\uselanguage{english}
\languagepath{english}

\usepackage[
  type={CC},
  modifier={by-nc-sa},
  version={4.0},
  lang={english},
]{doclicense}

\usefonttheme{professionalfonts}
% \usepackage[scaled]{beramono}
\usepackage[I]{sansmathfonts}
\usepackage[bb=boondox]{mathalfa}

\usepackage{oststud}
\usepackage{regtec}
\usepackage{tikz}
\usetikzlibrary{matrix}
\usetikzlibrary{math}
\usetikzlibrary{shapes.geometric}
\usetikzlibrary{fadings}

\usepackage{pgfplots}
\pgfplotsset{compat=1.18}

% source code
\usepackage{listings}
%% create a lstlisting style
\lstdefinestyle{samplestyle}{
    belowcaptionskip=\baselineskip,
    breaklines=true,
    frame=none,
    inputencoding=utf8,
    % margin
    xleftmargin=\parindent,
    % background
    backgroundcolor=\color{hsr-lightgrey20},
    % default language:
    language=[LaTeX]TeX,
    showstringspaces=false,
    % font
    basicstyle=\ttfamily\small,
    identifierstyle=\color{hsr-black},
    keywordstyle=\color{hsr-blue},
    commentstyle=\color{hsr-black40},
    stringstyle=\color{hsr-mauve80},
}

%% and set the chosen style
\lstset{style=samplestyle, escapechar=`}

\title{Computing Trajectories for Vertical Landing}
\subtitle{\slshape Computational Control Project}
\author{Naoki Sean Pross}
\date{Spring Semester 2023}
\institute[ETHZ]{ETH Zürich}

\begin{document}

\frame{
  \maketitle
  \centering
  \doclicenseImage[imagewidth=5em]
}

\section{Current State}

\begin{frame}[fragile]{Current State}
  \begin{columns}
    \begin{column}[b]{.5\linewidth}
      \begin{block}{Rocket Model}
        Non-linear dynamics linearised around $z_s = 0$,
        $u_s = \mt{\begin{bmatrix} mg & 0 & 0 \end{bmatrix}}$:
        \[
          z_{n+1} = A z_n + B u_n,
        \]
        where
        \begin{align*}
          z &= \mt{\begin{bmatrix}
            ~x & y & \dot{x} & \dot{y} & \theta & \dot{\theta}~
          \end{bmatrix}}, \\
          u &= \mt{\begin{bmatrix}
            ~F_E & F_S & \varphi~
          \end{bmatrix}}.
        \end{align*}
      \end{block}
      \begin{block}{Controller}
        Decoupled PID controllers for $F_E$, $F_S$ and $\varphi$,
        unaware of each other.
      \end{block}
    \end{column}
    \begin{column}[b]{.5\linewidth}
      \begin{center}
        \begin{tikzpicture}[
            dot/.style = {
              circle, fill=white, thick, draw=black, 
              inner sep = 0, outer sep = 0,
              minimum size = .8mm
            },
            force/.style = {thick, -latex},
          ]
          \begin{scope}[rotate=-30]
            \coordinate (C) at (0,0);

            \fill[lightgray!40] (-.25,-1) rectangle (.25,2);
            % nozzle
            \fill[lightgray!40] (-.1,-1) -- (.1,-1) --
              (.2,-1.2) -- (-.2,-1.2) -- cycle;
            % head
            \fill[lightgray!40] (-.25,2) -- (0,2.2) -- (.25,2) -- cycle;

            \draw[thick] (0,0) --
              node[midway, below right] {$\ell_1$} (0,2)
              coordinate (S);
            \draw[thick] (0,0) --
              node[midway, above left] {$\ell_2$} (0,-1)
              coordinate (E);

            \draw[force, blue!80!black] (S) --
              node[near end, above right] {$F_S$} ++(.75,0);
            \draw[force, blue!80!black] (E) --
              node[near end, above left] {$F_E$} ++(-120:1.6);

            \draw[thick, gray, densely dotted] (E) -- ++(0,-1.4)
              coordinate (Ee);
            \draw[thick, gray] ($(E)+(0,-.75)$) arc (-90:-120:.75)
              node[midway, below left] {$\varphi$};
          \end{scope}

          \draw[thick, gray, densely dotted] (C) -- ++(0,2);
          \draw[thick, gray, densely dotted] (C) -- ++(0,-2);

          \draw[force, red!80!black] (C) -- ++(0,-1)
            node[right] {$mg$};

          \draw[thick, gray] ($(C)+(0,-1.6)$) arc (-90:-120:1.6)
            node[midway, below] {$\theta$};

          \node[dot] at (C) {};
          \node[dot] at (E) {};
          \node[dot] at (S) {};

          \node[dot] (A) at ($(C)+(-1.7,1)$) {};
          \draw[thick, -latex] (A) -- ++ (.5,0) node[right] {$\uvec{x}$};
          \draw[thick, -latex] (A) -- ++ (0,.5) node[above] {$\uvec{y}$};
        \end{tikzpicture}
      \end{center}
      \begin{alertblock}{Behaviour}
        \begin{itemize}
          \item Work well for ``good'' $z_0$
          \item Breaks easily $\leadsto$ need to retune
          \item Waits and high thrust near end
        \end{itemize}
      \end{alertblock}
    \end{column}
  \end{columns}
\end{frame}

\section{Failure Mode}

\begin{frame}[fragile]{Failure Mode}

  Plots: Trajectories on the $xy$ plane, color is the $y$ velocity (red is fast).

  \begin{columns}[T]
    \pgfplotsset{
      every axis/.style = {
        axis lines = left,
        % width = 43mm,
        width = 1.1\linewidth,
        height = 5cm,
        xlabel = {Position $x$},
        ylabel = {Position $y$},
        xmin = 0, xmax = 30,
        ymin = 5, ymax = 26,
        colormap name = hot, % viridis,
        semithick, font = \tiny,
        % xtick = \empty, ytick = \empty,
        % xticklabels = \empty, yticklabels = \empty,
      }
    }

    \tikzset{
      target/.style = {
        thick, draw = black, fill = white,
        inner sep = 0, outer sep = 0,
        shape = diamond, minimum size = 1mm,
      },
    }

    \begin{column}{.33\linewidth}
      \begin{exampleblock}{Bad $x_0$ Coordinate}
        Overshoots landing pad \\[.5em]
        \begin{tikzpicture}
          \begin{axis}
            % \addplot[thick, mesh] table[x=x, y=y, meta=ydot] {./trajectories/pid-0.dat};
            \addplot[thick, mesh] table[x=x, y=y, meta=ydot] {./trajectories/pid-1.dat};
            \addplot[thick, mesh] table[x=x, y=y, meta=ydot] {./trajectories/pid-2.dat};
            \addplot[thick, mesh] table[x=x, y=y, meta=ydot] {./trajectories/pid-3.dat};
            \node[target, label={180:Target}] (T) at (axis cs:16.7, 7.5) {};
            \draw[lightgray, densely dotted] (T) -- ++(axis cs:0,23);
          \end{axis}
        \end{tikzpicture}
      \end{exampleblock}
    \end{column}

    \begin{column}{.33\linewidth}
      \begin{exampleblock}{Bad $\theta_0$ Angle}
        Not enough side thrust \\[.5em]
        \begin{tikzpicture}
          \begin{axis}
            \addplot[thick, mesh] table[x=x, y=y, meta=ydot] {./trajectories/pid-4.dat};
            \addplot[thick, mesh] table[x=x, y=y, meta=ydot] {./trajectories/pid-5.dat};
            \addplot[thick, mesh] table[x=x, y=y, meta=ydot] {./trajectories/pid-6.dat};
            \node[target, label={0:Target}] (T) at (axis cs:16.7, 7.5) {};
            \draw[lightgray, densely dotted] (T) -- ++(axis cs:0,23);
          \end{axis}
        \end{tikzpicture}
      \end{exampleblock}
    \end{column}

    \begin{column}{.33\linewidth}
      \begin{exampleblock}{Bad $y_0$ Coordinate}
        Too little thrust \\[.5em]
        \begin{tikzpicture}
          \begin{axis}
            \addplot[thick, mesh] table[x=x, y=y, meta=ydot] {./trajectories/pid-7.dat};
            \addplot[thick, mesh] table[x=x, y=y, meta=ydot] {./trajectories/pid-8.dat};
            \addplot[thick, mesh] table[x=x, y=y, meta=ydot] {./trajectories/pid-9.dat};
            \node[target] (T) at (axis cs:16.7, 7.5) {};
          \end{axis}
        \end{tikzpicture}
      \end{exampleblock}
    \end{column}
  \end{columns}

  \begin{alertblock}{Intuition}
    Decoupled controllers cannot coordinate in difficult situations (far from set point) and fail hard.
  \end{alertblock}
\end{frame}

\section{Recommendation}

\begin{frame}[fragile]{Recommendation}
  \begin{columns}[T]
    \begin{column}{.5\linewidth}
      \begin{block}{Proposed Controller} \small
        Relaxed linear MPC on linearised dynamics \\[.5em]

        \textbf{Strengths}
        \begin{itemize}
          \item Cutting edge, yet proven to be reliable
          \item Optimize fuel consumption
          \item ``Easy'' to specify constraints
          \item Possible to extend with more powerful theory if necessary (eg.
            sequential convex programming)
        \end{itemize}

        \textbf{Weaknesses}
        \begin{itemize}
          \item Computationally more expensive
          \item No theoretical stability guarantee (because of linearisation)
        \end{itemize}
      \end{block}
    \end{column}
    \begin{column}{.5\linewidth}
      \begin{alertblock}{Key Idea of MPC} \small
        Continuously predict future to decide next action.
      \end{alertblock}
      \begin{center}
        \begin{tikzpicture}[
            dot/.style = {
              circle, fill=white, thick, draw=black, 
              inner sep = 0, outer sep = 0,
              minimum size = .8mm
            },
            rocket/.pic = {
              \fill[lightgray!50] (-.25,-1) rectangle (.25,2);
              \fill[lightgray!50] (-.1,-1) -- (.1,-1) --
                (.2,-1.2) -- (-.2,-1.2) -- cycle;
              \fill[lightgray!50] (-.25,2) -- (0,2.2) -- (.25,2) -- cycle;
            }
          ]
          % Rocket
          \coordinate (R) at (-2, 4);
          % Target
          \coordinate (T) at (0,0);

          \draw (R) pic[scale=.6, rotate=10] {rocket};

          \draw[thick, draw=white, double=black] (R) to[out=120, in=-20] 
            node[near start, left, text width = 16mm] {Real \\ Trajectory} 
            ++(-1.2,1.5) node[dot] (Rp) {};

          \draw[thick, opacity=.8, draw=white, double=red!80!black, path fading=south]
            (Rp) to[out=-20, in=90] coordinate[pos=.3] (PP) (T);
          \node[above right, text width = 15mm, red!80!black]
            at (PP) {Previous Prediction};

          \draw[thick, opacity=.8, draw=white, double=blue!80!black, path fading=south]
            (R) to[out=-60, in=90] coordinate[pos=.3] (CP) (T);
          \node[above right, fill = white, text width = 15mm, text = blue!80!black]
            at (CP) {Current Prediction};

          \node[dot] at (R) {};
          \node[dot, shape = diamond, minimum size = 1mm,
            label={180:Target}] at (T) {};

          \begin{scope}[shift={(-3.7,1)}, font=\small]
            \node [
              thick, draw=gray, fill=white,
              minimum width = 2.2cm, minimum height=3.5cm,
            ] (box) at (.1,.6) {};
            \draw pic[scale=.6] {rocket};
            
            \node[gray, anchor = south west, font=\small]
              at (box.north west) {Constraints};

            \fill[path fading=north, hsr-petrol, opacity=.8]
              (120:1.5) arc (120:60:1.5) -- (60:.5) arc (60:120:.5) -- cycle;

            % \draw[dashed] (0,0) -- (0,1.7);
            \draw[hsr-petrol] (0,1.7) arc (90:60:1.7) 
              node[near start, above] 
              {$\theta \in (\underline{\theta}, \overline{\theta})$};

            \draw[hsr-mauve, thick, -latex] (0,-.6)
              node[above, font=\small, fill=white!50, inner sep = 0, outer sep = 1mm]
              {$0 \leq F_E < \overline{F}_E$} -- ++(0,-.5);
          \end{scope}

          \draw[thick, gray, dashed, ->] (CP) to[out=-160, in=0]
            node[near end, anchor = north west, below right, font=\small, text width=2cm] 
              {Takes into \\ account} (box.east);
        \end{tikzpicture}
      \end{center}
    \end{column}
  \end{columns}
\end{frame}

\section{Demonstration}

\begin{frame}{Demonstration}
  Plots: Trajectories on the $xy$ plane, color is the $y$ velocity (red is fast).

  \begin{columns}[T]
    \pgfplotsset{
      every axis/.style = {
        axis lines = left,
        % width = 43mm,
        width = 1.1\linewidth,
        height = 5cm,
        xlabel = {Position $x$},
        ylabel = {Position $y$},
        xmin = 0, xmax = 30,
        ymin = 5, ymax = 26,
        colormap name = hot, % viridis,
        semithick, font = \tiny,
        % xtick = \empty, ytick = \empty,
        % xticklabels = \empty, yticklabels = \empty,
      }
    }

    \tikzset{
      target/.style = {
        thick, draw = black, fill = white,
        inner sep = 0, outer sep = 0,
        shape = diamond, minimum size = 1mm,
      },
    }

    \begin{column}{.33\linewidth}
      \begin{exampleblock}{Bad $x_0$ Coordinate}
        \begin{tikzpicture}
          \begin{axis}
            % \addplot[thick, mesh] table[x=x, y=y, meta=ydot] {./trajectories/pid-0.dat};
            \addplot[thick, mesh] table[x=x, y=y, meta=ydot] {./trajectories/mpc-1.dat};
            \addplot[thick, mesh] table[x=x, y=y, meta=ydot] {./trajectories/mpc-2.dat};
            \addplot[thick, mesh] table[x=x, y=y, meta=ydot] {./trajectories/mpc-3.dat};
            \node[target, label={0:Target}] (T) at (axis cs:16.7, 7.5) {};
            \draw[lightgray, densely dotted] (T) -- ++(axis cs:0,23);
          \end{axis}
        \end{tikzpicture}
      \end{exampleblock}
    \end{column}

    \begin{column}{.33\linewidth}
      \begin{exampleblock}{Bad $\theta_0$ Angle}
        \begin{tikzpicture}
          \begin{axis}
            \addplot[thick, mesh] table[x=x, y=y, meta=ydot] {./trajectories/mpc-4.dat};
            \addplot[thick, mesh] table[x=x, y=y, meta=ydot] {./trajectories/mpc-5.dat};
            \addplot[thick, mesh] table[x=x, y=y, meta=ydot] {./trajectories/mpc-6.dat};
            \node[target, label={0:Target}] (T) at (axis cs:16.7, 7.5) {};
            \draw[lightgray, densely dotted] (T) -- ++(axis cs:0,23);
          \end{axis}
        \end{tikzpicture}
      \end{exampleblock}
    \end{column}

    \begin{column}{.33\linewidth}
      \begin{exampleblock}{Bad $y_0$ Coordinate}
        \begin{tikzpicture}
          \begin{axis}
            \addplot[thick, mesh] table[x=x, y=y, meta=ydot] {./trajectories/mpc-7.dat};
            \addplot[thick, mesh] table[x=x, y=y, meta=ydot] {./trajectories/mpc-8.dat};
            \addplot[thick, mesh] table[x=x, y=y, meta=ydot] {./trajectories/mpc-9.dat};
            \node[target, label={0:Target}] (T) at (axis cs:16.7, 7.5) {};
          \end{axis}
        \end{tikzpicture}
      \end{exampleblock}
    \end{column}
  \end{columns}

  \begin{columns}[t]
    \begin{column}{.5\linewidth}
      \begin{block}{Trajectories}
        MPC handles all situation where PID failed, because it is ``aware'' of
        what the other actuators are doing.
      \end{block}
    \end{column}
    \begin{column}{.5\linewidth}
      \begin{alertblock}{Note}
        Performance does not come for free: it is computationally (a lot) more
        expensive, but worth it!
      \end{alertblock}
    \end{column}
  \end{columns}
\end{frame}

\begin{frame}[fragile]{Deployment Plan}
  \begin{columns}[T]
    \begin{column}{.55\linewidth}
      \begin{tikzpicture}
        \begin{axis}[
              semithick,
              width = \linewidth,
              xlabel = {Sampling time / ms},
              ylabel = {Solve time / CPU cycles},
              ybar, % bar width = 7pt,
              symbolic x coords = {250, 100, 50, 20, 10},
              xtick = data, ymax = 2e8,
              enlarge y limits = 0.05,
              enlarge x limits = 0.15,
              font = \small,
              legend style = {
                at = {(1, 1.05)},
                anchor = south east,
                legend columns = -1,
                draw = none,
              },
              axis on top,
          ]

          \coordinate (X) at (axis cs: 50,0);
          \fill[draw = none, left color = white, right color = hsr-mauve40] 
            (X |- current axis.south) |- (current axis.north east) |- cycle;

          \addplot+[
            hsr-mauve, sharp plot, dashed, thick, fill = none,
            update limits = false, mark = -,
            every mark/.append style = { solid },
          ] table [
            x = sampletime-ms, y = sampletime-cycles,
          ] {./mpc-solvetimes.dat};
          \addlegendentry{Bound}

          \addplot+[
            hsr-blue, fill = hsr-blue40,
            semithick, nodes near coords,
            nodes near coords style = {
              above = 1mm,
              fill = white, 
              fill opacity = .8,
              text opacity = 1,
              inner sep = 1,
            },
            error bars/.cd,
              y dir = both,
              y explicit,
          ] table [
            x = sampletime-ms, y = cycles-mean, y error=cycles-se,
            meta=horizon, point meta=explicit symbolic,
          ] {./mpc-solvetimes.dat};
          \addlegendentry{MPC}

          \node[hsr-blue, text width = 2cm, font = \small] (T) 
            at (axis cs: 100, 1e8) {Horizon length in samples};
          \draw[hsr-blue, thick, ->] (T.south) to[out=-90, in=90] (axis cs: 100, .4e8);
        \end{axis}
      \end{tikzpicture}
      {\small
        Plot: CVXPY with time horizon of 10~s.
      }
      \begin{alertblock}{Hardware}
        \small
        Modern hardware is very powerful.
        Decision factors are sampling time and prediction time horizon.
      \end{alertblock}
    \end{column}
    \begin{column}{.45\linewidth}
      \begin{block}{Computation}
        \small
        CPU cycles\footnote{Computation time normalized wrt CPU freq. Plot $f =
        3.22$~GHz.} needed to predict fixed amount of time into the future grows
        exponentially with the sampling frequency. Solve time is bounded
        by sampling time (need action before next sample comes).
      \end{block}
      \begin{exampleblock}{Solver Software}
        \small
        There are countless options: \\[.2em]
        \textbf{Commercial solutions}
        \begin{itemize}
          \item Embotech AG, MOSEK ApS
        \end{itemize}
        \textbf{Free solutions}
        \begin{itemize}
          \item CVXgen, CVXPYgen, OSQP, OOQP, CVXOPT, ECOS
        \end{itemize}
      \end{exampleblock}
    \end{column}
  \end{columns}
\end{frame}

\section{Model-Free Recommendation}

\iffalse
\begin{frame}{Model-Free Recommendation}
  \begin{columns}
    \begin{column}{.5\linewidth}
      \begin{alertblock}{Model Uncertainty}
        Linearised model is very inaccurate in $x$ and $\theta$!
      \end{alertblock}
    \end{column}
    \begin{column}{.5\linewidth}
    \end{column}
  \end{columns}
\end{frame}
\fi

\appendix

\frame{
  \centering
  {\Huge\bfseries Backup Slides} \\[1em]
  If someone wants to know the details \\
  \textcolor{hsr-lightgrey}{(they are not officially part of the presentation)}
}

\begin{frame}{What is Relaxed Linear MPC}
  \begin{block}{Relaxed Linear MPC} \small
    Non-linear dynamics linearised at $(z_s, u_s)$ to get LTI system $(A, B)$,
    target landing pad is at $z_f$. In state $z_n$ compute
    \begin{align*}
      u^\star - u_s = \argmin_{u_0} &\Bigg\{
        \mt{z}_N S z_N + \sum_{k=0}^{N-1}
          \mt{z}_k Q z_k + \mt{u}_k R u_k + V \|\epsilon_k\|_1 
        \Bigg\} \\
      \text{subject to}\quad & 
        \begin{aligned}[t]
          z_{k+1} &= A z_k + B u_k  && \text{(dynamics)} \\
          G_z z_k &\leq g_z - G_z z_s + \epsilon_k && \text{(relaxed state constr.)} \\
          G_u u_k &\leq g_u - G_u u_s && \text{(input constr.)}\\
          z_N &= z_f - z_s && \text{(terminal constr.)} \\
          z_0 &= z_n - z_s && \text{(parametrisation)}
        \end{aligned}
    \end{align*}
    Index $n$ is real time, $k$ is the prediction time. The $\epsilon_k$ are
    linearly penalized slack variables, and $N$ is the ``horizon length'' for
    the prediction.
  \end{block}

  \begin{alertblock}{Model Uncertainty} \small
    The linearised model is very inaccurate in $x$ and $\theta$. To take
    into account make future states more expensive:
    \(
      Q_k = \operatorname{diag} \begin{bmatrix}
        q_0 + \varsigma_0 k/N & \dots
        & q_{n_x} + \varsigma_{n_x} k/N
      \end{bmatrix}.
    \)
  \end{alertblock}
\end{frame}

\end{document}
